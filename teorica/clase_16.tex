\section{Clase 16}

\begin{definition}
Sea $P$ un programa de lenguaje $\mathcal{S}$. Una ecuaci\'on es una identidad $V = m$, donde $V$ es una variable y $m \in \mathbb{N}$. Un estado de $P$ es un conjunto de ecuaciones sujeto a la siguiente condici\'on:
 
\begin{enumerate}
	\item No puede existir en dicho conjunto 2 ecuaciones del tipo $V = m$ y $V = m'$ con $m = m'$.
	\item Si $V$ es una variable que aparece en $P$, dicho conjunto debe contener $m \neq m'$ en una ecuaci\'on del tipo $V = m$.
	\item una \textbf{descripci\'on instant\'anea} de P es un par $(i, \tau)$ donde $1 \leq i \leq n + 1$	y $\tau$ es un estado.
	\item Si $1 \leq i \leq n$ y $(i, \tau)$ es una descripci\'on instant\'anea, se define sucedo de $(i, \tau)$ a la siguiente descripci\'on instant\'anea: suc$(i, \tau)$ = $(j, \sigma)$ del siguiente modo:
	\begin{enumerate}
		\item si $I_{i}$ es $V \leftarrow V + 1$, $j = i + 1$ y en $\tau$ el estado es el mismo que $\tau$ salvo que la ecuaci\'on $V = m$ se transforma en $V = m + 1$.
		\item Si $I_{i}$ es $V \leftarrow V - 1$, la ecuaci\'on $V = m$ se transforma en $\tau$ en la ecuaci\'on $V = m - 1$, si $m > 0$ y si $m = 0$, $\sigma = \tau$ y en ambos casos $j = i + 1$
		\item Si $I_{i}$ es la instrucci\'on $IF V \neq 0 GOTO \; L$, en este caso $\tau = \sigma$ y $j$ se define del siguiente modo:
		\begin{enumerate}
			\item Si $V = 0 \in \tau$, $j = i + 1$, si $V = m \in \tau$ y $m \neq 0$.
			\item Si $V = m \in \tau$ y $m \neq 0$. Si en $P$ no hay ninguna instrucci\'on con etiqueta $L$, $j = n + 1$. Si en $P$ aparece al menos una instrucci\'on con etiqueta $L$, $j$ es el menor \'indice $i \ L$, aparece en $I_{i}$
		\end{enumerate}
	\end{enumerate}
\end{enumerate}
\end{definition}
 
\begin{definition}
Un \textbf{c\'omputo} de $P$ es una secuencia $S_1, S_2, \ldots, S_k$ de descripci\'on instant\'aneas tal que $S_{i + 1} = suc(S)$
\end{definition}
 
\begin{definition}
Sea $P$ un programa del lenguaje $\mathcal{S}$ y sean $I_1, I_2, \ldots, I_n$ las instrucciones de $P$. Si $m > 0$, definimos la funci\'on $\varPsi_{p}^{m}(a_1, a_2, \ldots, a_m)$. Fijamos el siguiente estado inicial $\sigma$: $X_1 = a_1, X_2 = a_2, \ldots, X_m = a_m$ y $V = 0$ para toda variable $V$ diferente de $X_1, X_2, \ldots, X_m$ y la descripci\'on instant\'anea inicial es para $(1, \sigma)$.
 
Hay dos casos posibles:
	\begin{enumerate}
		\item Existe un c\'omputo $S_{1}, S_{2}, \ldots, S_{k}$ donde $S_{k}$ es la descripci\'on instant\'anea terminal. $S_{k} = (n + 1, \tau)$.
		\item No existe c\'omputo \textbf{finito}. En este caso $\varPsi_{p}^{m}(a_1, a_2, \ldots, a_m)$
	\end{enumerate}
\end{definition}
 
\begin{example}
$Y \leftarrow Y + 1$
 
$Y \leftarrow Y + 1$
 
$Y \leftarrow Y + 1$
 
$\ldots$
 
$Y \leftarrow Y + 1$
 
$\varPsi_{p}^{m}(a_1, a_2, \ldots, a_m) = k$
\end{example}
 
\begin{example}
$IF X_1 = 0 GOTO E$
 
$[A] X_1 \leftarrow X_1 - 1$
 
$Y \leftarrow Y + 1$
 
$IF X \neq 0 GOTO \; A$
 
$\varPsi_{p}^{1}(x) = x$
 
$\varPsi_{p}^{2}(x_1, x_2) = x_1$
 
Y en general:
 
$\varPsi_{p}^{m}(x_1, x_2, \ldots, x_m) = x_1$
\end{example}
 
\begin{example}
Sea $P$ un programa que computa la funci\'on suma: donde $X_1$ y $X_2$ son las variables de entrada del programa $P$
 
$\varPsi_{p}^{2}(x_1, x_2) = x_1 + x_2$
 
$\varPsi_{p}^{1}(x_1) = x_1 + 0 = x_1$
 
\end{example}
 
\begin{example}
Sea $P$ el siguiente programa:
 
$[A] IF X_1 \neq 0 \; GOTO A$
 
$\varPsi_{p}^{1}(3) = \uparrow$
 
$(1, X_1 = 3) = (1, X_1 = 3) = (1, X_1 = 3) = \ldots = (1, X_1 = 3)$
 
$\varPsi_{p}^{0}(0) = 0$
 
$(1, X_1 = 0), (2, X_1 = 0, Y = 0)$
\end{example}
 
\begin{definition}
Sea $\mathcal{F}$ una funci\'on de k-variables, Diremos que $\mathcal{F}$ es \textbf{total}, si el dominio de $\mathcal{F}$ es $\mathbb{N}^{k} = \mathbb{N} \times \ldots \times \mathbb{N}$. Diremos que $\mathcal{F}$ es \textbf{parcial} si el dominio de $\mathcal{F}$ es un subconjunto de $\mathcal{N}^{k}$.
\end{definition}
 
\begin{example}
$\mathcal{F}(x_1, x_2) = x_1 - x_2$. $\mathcal{F}$ es parcial pero no \textbf{total}.
\end{example}
 
\begin{definition}
Diremos que $\mathcal{F}$ es \textbf{parcialmente computable} si existe un programa $P$ tal que $\varPsi_{p}^{k} = \mathcal{F}$
\end{definition}
 
\begin{definition}
\textbf{Composicion}: Sea $\mathcal{F} : \mathbb{N}^{k} \rightarrow \mathbb{N}$ y sea $\mathcal{G}: \mathbb{N}^{n} \rightarrow \mathbb{N}$ y $g_2 \colon \mathbb{N}^{n} \rightarrow \mathbb{N}$, $\ldots$, $g_k \colon \mathbb{N}^{n} \rightarrow \mathbb{N}$, k funciones de n variables. Se define la composici\'on de $\mathcal{F}$ con las funciones $g_i 1 \leq i \leq k$ a la funcion $h(x_1, x_2, \ldots, x_n) = \mathcal{F}(g(x_1, x_2, \ldots, x_n), g_2(x_1, x_2, \ldots, x_n), \ldots, g_k(x_1, x_2, \ldots, x_n))$
\end{definition}
 
\begin{theorem}
Sea $S_i, F, g_1, g_2, \ldots, g_n$ son funciones computables, entonces h es una funci\'on computable
\end{theorem}
 
\begin{proof}[Demostraci\'on]
El siguiente programa computa h:
 
$Z_1 \leftarrow g_1(x_1, X_2, \ldots, x_n)$
 
$Z_2 \leftarrow g_1(x_1, X_2, \ldots, x_n)$
 
$Z_3 \leftarrow g_1(x_1, X_2, \ldots, x_n)$
 
$Z_k \leftarrow g_1(x_1, X_2, \ldots, x_n)$
 
$Y \leftarrow F(Z_1, Z_2, \ldots, Z_k)$
\end{proof}
 
\begin{definition}
\textbf{Recursion} Sea $H \colon \mathbb{N}^{m + 2} \rightarrow \mathbb{N}$ una funci\'on de $m + 2$ variables y $g \colon \mathbb{N}^{m} \rightarrow \mathbb{N}$. Se define la siguiente funci\'on $F \colon \mathbb{N}^{m + 1} \rightarrow \mathbb{N}$ como sigue:
 
\begin{enumerate}
	\item $F(x_1, x_2, \ldots, x_m, 0) = g(x_1, x_2, \ldots, x_m)$
	\item $F(x_1, x_2, \ldots, x_m, t + 1) = H(x_1, x_2, \ldots, x_m, t, F(x_1, x_2, \ldots, x_m, t))$
\end{enumerate}
 
Si hacemos un seguimiento, $F(1) = H(0, k)$, $F(2) = H(1, F(1))$ $\ldots$
 
\end{definition}
 
\begin{theorem}
Si H es \textbf{computable} y g es \textbf{computable}, entonces F es \textbf{computable}
\end{theorem}
 
\begin{proof}[Demostraci\'on]
El siguiente programa computa a $F$
 
$Y \leftarrow k$
 
$[A] IF X = 0 \; GOTO \; E$
 
$Y \leftarrow H(Z, Y)$
 
$Z \leftarrow Z + 1$
 
$X \leftarrow X - 1$
 
$GOTO A$
\end{proof}
 
\begin{example}
Si $n \in \mathbb{N}$, n! se define: $0! = 1$ y $(n + 1) = n! \times (n + 1)$
 
Aqu\'i $k = 1$
 
$F(n) = n!$
 
$F(n + 1) = (n + 1)! = H(n, \underbrace{n!}_{m}) = \underbrace{n!}_{m}\times(n + 1) \rightarrow H(n, m) = m \times (n + 1) = m \times n + m$
\end{example}
 
\begin{definition}
Una funci\'on de k variables se dice \textbf{recursiva primitiva} si se la obtiene en un n\'umero finito de pasos a partir de las \textbf{proyecciones}, la \textbf{funci\'on constante 0} y la \textbf{funci\'on sucesora}, utilizando composici\'on y/o recursi\'on de alguno de los tipos anteriores.
\end{definition}
 
Proyecciones:
 
$\varphi_i : \mathbb{N}^k \rightarrow \mathbb{N}$
 
$\varphi_i (x_1, x_2, \ldots, x_k) = x_i$
 
Funci\'on constante cero:
 
$cero(x) = 0$
 
Sucesor
 
$suc(x) = x + 1$
 
Suma
 
$suma(x_1, x_2) = x_1 + x_2$
 
$suma(x_1, 0) = x_1$
 
$suma (x_1, t + 1) = x_1 + (t + 1) = (x_1 + t) + 1 = suma(x_1, t) + 1 = H(x_1, t, suma(x_1, t))$
 
\begin{colorario}
Toda funci\'on recursiva primitiva es \textbf{computable}
\end{colorario}