\section{Clase 11}

Un conjunto de f\'ormulas $T$ se dice de Godel si es completo, consistente, cerrado por las reglas de inferencia y $A \times \mathcal{L} \subseteq T$ (axiomas l\'ogicos).

\subsection{Teorema de Completitud de Godel}

\begin{enumerate}
	\item Si $T$ es un conjunto de Godel entonces existe una interpretaci\'on $\mathcal{I}$ tal que $T_{I}$
	\item Si $S$ es un conjunto de f\'ormulas consistentes y tal que $S \subseteq T$
\end{enumerate}

\begin{colorario}
Sea $\mathcal{L}$ de primer orden y sea $E$ un conjunto de f\'ormulas de $\mathcal{L}$. Si $\epsilon$ es una f\'ormula arbitraria, las siguientes condiciones son equivalentes:

\begin{enumerate}
	\item[a)] $T \vdash \epsilon$
	\item[b)] $T$ es inconsistente o $T \models \epsilon$
\end{enumerate}
\end{colorario}

\begin{proof}
$a \rightarrow b$: ya lo hicimos

$b \rightarrow a$:

Si $T$ es inconsistente $T \vdash \epsilon$ $\forall \epsilon \in Form$ (de un conjunto inconsistente se puede probar cualquier f\'ormula)

Asumimos que $T$ es consistente (por hip\'otesis). $T \models \epsilon$ podemos suponer, sin p\'erdida de generalidad, que $A \times \mathcal{L} \subseteq T$. Sea $\bar{\epsilon}$ la clausura de $\epsilon$, $\bar{\epsilon} = \forall x_{j_1}, \ldots, \forall x_{j_r}$ donde $x_{j_1}, \ldots, x_{j_r}$ son las variables libres de $\epsilon$. Es claro que $T \models \bar{\epsilon}$. Tenemos dos casos:

\begin{enumerate}
	\item[I)] $T \cup \{ \neg \bar{\epsilon}\}$ es inconsistente.
	\item[II)] $T \cup \{ \neg \bar{\epsilon}\}$ es consistente.
\end{enumerate}


I. De un conjunto inconsistente puedo deducir cualquier f\'ormula entonces $T \cup \{ \neg \bar{\epsilon}\} \vdash \bar{\epsilon}$. Por el teorema de la deducci\'on obtenemos que $T \vdash (\neg \bar{\epsilon} \rightarrow \bar{\epsilon})$ sabemos que $T \vdash (\neg \bar{\epsilon} \rightarrow \neg \bar{\epsilon}) \rightarrow ((\neg \bar{\epsilon} \rightarrow \bar{\epsilon}) \rightarrow \bar{\epsilon})$ (axioma 3), como $T \vdash (\neg \bar{\epsilon} \rightarrow \bar{\epsilon})$ entonces usando M.P. $T \vdash ((\neg \epsilon \rightarrow \bar{\epsilon}) \rightarrow \bar{\epsilon})$ y usando de vuelta el M.P. $T \vdash \bar{\epsilon}$. Usando el Axioma 5, $(\forall x_j \; \beta \rightarrow \beta)$ para las variables $x_{j_1}, \ldots, x_{j_r}$ se obtiene $T \vdash \epsilon$

II. Como $T \cup \{ \neg \bar{\epsilon}\}$ es consistente, por el teorema de completitud, existe una interpretaci\'on $\mathcal{I}$ que satisface a $T$ y a $\neg \bar{\epsilon}$, como $T \models \epsilon$ entonces $\mathcal{I}$ satisface a $\epsilon$ y a $\neg \bar{\epsilon} = \neg \forall x_{j_1}, x_{j_2}, \ldots, x_{j_r} \epsilon$  lo que es imposible

\end{proof}

\begin{colorario}
\textbf{Teorema de Completitud}. Si $\mathcal{L}$ es un lenguaje de primer orden y $T$ es un conjunto de f\'ormulas. Entonces se verifica la siguiente condici\'on:

Si $T \models \epsilon$ entonces existe $\hat{T} \subseteq T$ tal que $\hat{T}$ es finito y $\hat{T} \models \epsilon$.
\end{colorario}

\begin{proof}
Se puede asumir sin p\'erdida de generalidad que $\mathcal{L} \times A \subseteq T$. Por el colorario anterior se tiene que $T \vdash \epsilon$ y luego existe $\hat{T} \subseteq T$ finito tal que $\hat{T}\vdash \epsilon$ (pues una demostraci\'on tiene finita cantidad de eslabones) y luego $\hat{T} \models \epsilon$
\end{proof}

\begin{aplicacion}
No se puede expresar en un lenguaje con igualdad que un conjunto sea finito:

1 elemento: $\varepsilon_1 = \forall x_1 \; x_1 = x_2 \rightarrow \bar{\varepsilon} = \forall x_1 \forall x_2 \; x_1 = x_2$

2 elemento: $\varepsilon_2 = \exists x_1 \exists x_2 (\neg x_1 = x_2 \land \forall (x_3 = x_1 \land x_3 = x_2))$

$\vdots$

n elementos: $\varepsilon_n = \exists x_1 \exists x_2 \ldots \exists x_n (\bigwedge_{1 \leq i \leq n ; 1 \leq j \leq n; i \neq j} \neg x_i = x_j \land \forall x_m (\ldots))$

\end{aplicacion}

\begin{observation}
No puedo unir todos estos con un $\land$ para expresar que un conjunto es finito porque un enunciado tiene que tener finitas variables.
\end{observation}

\begin{proof}
Supongamos que por el absurdo que se puede expresar por un enunciado $\varepsilon$ que un conjunto es finito y $\varepsilon$ es un enunciado de $\mathcal{L}$

\begin{enumerate}
	\item[$\bullet$] $(\varepsilon \land \neg \varepsilon_1)$ es satisfacible. Cualquier universo con 2 elementos por ejemplo.
	\item[$\bullet$] $(\varepsilon \land \neg \varepsilon_1 \land \neg \varepsilon_2)$	es satisfacible. Cualquier universo con 3 elementos por ejemplo.
\end{enumerate}

En general para cada $n \geq 1$, el conjunto $\{ \varepsilon, \neg \varepsilon_1, \ldots \neg \varepsilon_n\}$ es satisfacible con el siguiente conjunto $\Gamma = \{ \neg \varepsilon_1, \ldots, $ $\neg \varepsilon_n, \ldots \}$ $\models \neg \varepsilon$. Por el teorema de la compacidad, $\neg \varepsilon$ es consecuencia de un n\'umero finito de elementos de $\Gamma$, es decir $\{ \neg \varepsilon_1, \ldots, \neg	\varepsilon_m \} \models \neg \varepsilon$  para cierto $m \geq 1$ lo que es imposible que viene de la suposici\'on hecha.

\end{proof}

\begin{definition}
Sea $\mathcal{L}$ un lenguaje de primer orden y sea $T$ un conjunto de f\'ormulas. Diremos que $\mathcal{L}$ es \textbf{suficiente} para $T$ si se verifica la siguiente condici\'on: si $\varepsilon$ es una f\'ormula de $\mathcal{L}$ tal que $\neg \forall x_j \; \varepsilon(x_j)$ es un enunciado de $T$ entonces existe un s\'imbolo de constante $c$ tal que:

$(\neg \forall x_j \; \varepsilon(x_j) \rightarrow \neg \varepsilon(c)) \in T$. Esta expresi\'on se puede transformar utilizando la siguiente propiedad: $\neg \forall x_j \; \neg \varepsilon \equiv \exists x_j \; \varepsilon$ se puede escribir como: $(\exists x_j \; \beta(x_j) \rightarrow \beta(c))\in T$
\end{definition}