\section{Clase 18}
 
\subsection{Aplicaciones}
 
\begin{example}
El predicado de divisibilidad es primitivo recursivo:
 
\[
x | y =
\begin{cases}
1			&\text{si $x$ es divisor de $y$}		\\
0			&\text{sino}							\\
\end{cases}
\]
 
$x | y$ si y s\'olo si existe un $t \in \mathbb{N}$ tal que $y = x \times t$ $\Rightarrow$ $x | y$ si y s\'olo si existe un $t \leq y$ tal que $y = x \times t$ (para que quede acotado).
 
$x | y$ coincide con el predicado $\exists_{t \leq y} (y = x \times t)$. El predicado es primitivo recursivo por ser composici\'on de funciones como el producto y el existencial acotado es primitivo recursivo.
\end{example}
 
\begin{example}
Sea $\theta \colon \mathbb{N} \leftarrow \mathbb{N}$ dada por $\theta(0) = 0$ y si $n > 0$, $\theta(n) = \bullet | \alpha$ en donde $\alpha$ es cada uno de los divisores de $n$.
 
Si $n = p_{1}^{\alpha_1} \times \ldots \times p_{k}^{\alpha_k}$ $\theta(n) = (\alpha_1 + 1) \times (\alpha_2 + 1) \times \ldots \times (\alpha_k + 1)$.
 
$\theta(n) = \sum_{i = 1}^{n} (i | n)$ (suma acotada por n la entrada de la funci\'on)
 
$\theta(n)$ es una suma acotada de un predicado recursivo primitivo el cual es primitivo recursivo. Por lo tanto $\theta$ es recursiva primitiva.
 
\end{example}
 
\begin{example}
El siguiente predicado es primitivo recursivo:
 
\[
primo(n) =
\begin{cases}
1			&\text{si $n$ es primo}		\\
0			&\text{sino}				\\
\end{cases}
\]
 
\[
primo(n) =
\begin{cases}
1			&\text{si $\theta(n) = 2$}			\\
0			&\text{si $\theta(n) \neq 2$}		\\
\end{cases}
\]
 
primo es recursivo primitivo ya que $\theta$ es recursivo primitivo, $=$ es recursivo primitivo y $2$ es recursivo primitivo.
\end{example}
 
\begin{definition}[Minimizaci\'on Acotada]
Sea $P(x_1, x_2, \ldots, x_n, t)$ un predicado de $n + 1$ variables. Definimos la \textbf{minimizaci\'on} asociado al predicado $P$ a la siguiente funci\'on de $n$ variables es verdadera:
 
\[
\mathcal{F}(x_1, x_2, \ldots, x_n) = min_{t}\;P(x_1, x_2, \ldots, x_n, t)
\]
 
$\mathcal{F}(x_1, \ldots, x_n)$ \textbf{no est\'a definido} cuando no existe $t \ P(x_1, \ldots, x_n, t)$ es verdadero.
\end{definition}
 
\begin{example}
$P(x, t) \colon x < t$
 
$\mathcal{F}(x) = min_{t} x < t$
 
$\mathcal{F}(x) = x + 1$
 
\end{example}
 
\begin{example}
Si $Q(x, t) \colon t < x$
 
$\mathcal{F}(x) = min_{t} t < x$
 
$\mathcal{F}(0) = \uparrow$ y $\mathcal{F}(x) = x - 1$ si $x > 0$
 
\end{example}
 
\begin{proposition}
Si $P(x_1, \ldots, x_n, t)$ es un predicado computable de $n + 1$ variables entonces $min_{t}\;P(x_1, x_2, \ldots, x_n, t)$ es una funci\'on parcialmente computable
\end{proposition}
 
\begin{proof}[Demostraci\'on]
El siguiente programa computa el m\'inimo:
 
$[A]\;IF\;P(x_1, \ldots, x_n, Z) \neq 0 \; GOTO \; E$
 
$Z \leftarrow Z + 1$
 
$GOTO A$
 
$[E]\; Y \leftarrow Z$
\end{proof}
 
\begin{definition}[M\'inimo Acotado]
Sea $P(x_1, x_2, \ldots, x_n, t)$ un predicado de $n + 1$ variables. Se define $min_{t \leq y}\;P(x_1, \ldots, x_n, t)$. Como el menor $t \leq y / P(x_1, x_2, \ldots, x_n, t)$ es verdadero y toma el valor $0$ si no existe $t \leq y / P(x_1, x_2, \ldots, x_n, t)$
\end{definition}
 
\begin{proposition}
Si $P(x_1, \ldots, x_n, t)$ es un predicado recursivo primitivo de $n + 1$ variables entonces el $min_{t \leq y}$ $P(x_1, \ldots, x_n, t)$ es recursiva primitiva en las variables $x_1, \ldots, x_n, y$
\end{proposition}
 
\begin{proof}[Demostraci\'on]
Hacemos recursi\'on sobre $y$. Sea $\mathcal{F}(x_1, \ldots, x_n, y) = min_{t \leq y}\; P(x_1, \ldots, x_n, t)$.
 
\begin{enumerate}
	\item[] $\mathcal{F}(x_1, \ldots, x_n, 0) = 0$
	\item[] $\mathcal{F}(x_1, \ldots, x_n, y + 1)$
		\begin{enumerate}
			\item[] $\mathcal{F}(x_1, \ldots, x_n, y)$ si $\exists_{t \leq y}\;P(x_1, \ldots, x_n, t)$
			\item[] $y + 1$ si $\neg \exists_{t \leq y}\;P(x_1, \ldots, x_n, t) \land P(x_1, \ldots, x_n, y + 1)$			
			\item[] $0$ en otro caso
		\end{enumerate}
\end{enumerate}
 
$H(x_1, x_2, \ldots, x_n, y, \underbrace{\mathcal{F}(x_1, \ldots, x_n, y)}_{z})$
 
\[
H(x_1, x_2, \ldots, x_n, y, z) =
\begin{cases}
z			&		\\
y + 1		& 		\\
0			&
\end{cases}
\]
\end{proof}
 
\begin{example}
Sea $\mathcal{F} \colon \mathbb{N} \rightarrow \mathbb{N}$ la funci\'on $\mathcal{F}(x) = [\sqrt{x}]$. En donde [a] es el entero m\'as pr\'oximo a la izquierda de a:
 
[a] $\leq a < $ [a] + 1
 
$[\sqrt{x}] = min_{t \leq x}\;t + 1 \leq [\sqrt{x}] = min_{t \leq x} \; (t + 1)^2 > x$
\end{example}
 
\begin{example}
Sea $\mathcal{F} \colon \mathbb{N} \rightarrow \mathbb{N}$ dado por $\mathcal{F}(n) = P_n$ (n \'esimo m\'inimo primo). Entonces $\mathcal{F}$ es recursivo primitivo si $n \geq 1$.
 
\begin{enumerate}
	\item[] $\mathcal{F}(0) = 1$
	\item[] $\mathcal{F}(1) = 2$
	\item[] $\mathcal{F}(2) = 3$
	\item[] $\mathcal{F}(3) = 5$
\end{enumerate}
 
Definici\'on recursiva de $\mathcal{F}$.
 
\begin{enumerate}
	\item[]$\mathcal{F}(0) = 1$
	\item[]$\mathcal{F}(n + 1) = P_{n + 1} = min_{t \leq P_n! + 1}\;(primo(t) \land t > P_n)$
\end{enumerate}
 
Debemos buscar la funci\'on $H$, $H(n, \underbrace{\mathcal{F}(n)}_{P_n})$ $=$ $H(n, m) = min_{t \leq m! + 1} (primo(t) \land t > m)$
\end{example}
 
 
