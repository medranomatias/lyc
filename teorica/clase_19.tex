\section{Clase 19}
 
\subsection{Dos funciones para codificar programas en S}
\hfill
\begin{enumerate}
	\item Funci\'on Par
	\item N\'umero de Godel de una Secuencia
\end{enumerate}
 
\textbf{Funci\'on Par:} Es una funci\'on $<, > \colon \mathbb{N} \times \mathbb{N} \to \mathbb{N}$ biyectiva definida del siguiente modo:
 
\[
	<x, y> = 2^x \cdot (2 \cdot y + 1) - 1
\]
 
Veamos que $<, >$ es biyectiva
 
\begin{proof}[Demostraci\'on]
\hfill
\begin{enumerate}
	\item $<x, y> = <w, z>$ $\Rightarrow$ $2^x \cdot (2 \cdot y + 1) - 1 = 2^w \cdot (2 \cdot z + 1) - 1$ $\Rightarrow$ $2^x \cdot (2 \cdot y + 1) = 2^w \cdot (2 \cdot z + 1)$
		\begin{enumerate}
			\item[] si $x > w$ $\Rightarrow$ $2^{x - w} \cdot (2 \cdot y + 1) = 2 \cdot z + 1$ es absurdo, pues $2 \cdot z + 1$ es impar.
			\item[] si $w > x$ se llega a la misma contradicci\'on.
			\item[] si $w = x$ $\Rightarrow$ $2 \cdot y + 1 = 2 \cdot z + 1$ $\Rightarrow$ $y = z$ y por ende $<, >$ es biyectiva
		\end{enumerate}
	\item Sea $n \in \mathbb{N}$. Veamos que existen $x, y \in \mathbb{N}$ tales que $2^x \cdot (2 \cdot y + 1) - 1 = n$ o $2^x \cdot (2 \cdot y + 1) - 1 = n + 1$ $x = max \; \{ k : 2^k  \mid n+ 1 \}$ y $2 \cdot y + 1 = \cfrac{n + 1}{2^x}$ es impar $\Rightarrow$ $\cfrac{n + 1}{2^x} - 1$ es par y $y = \cfrac{\cfrac{n + 1}{2^x} - 1}{2}$ esto prueba que $<, >$ es biyectiva.
\end{enumerate}
\end{proof}
 
\begin{observation}
Notar que $<, >$ es recursiva primitiva: la funci\'on inversa ${<, >}^{- 1} \colon \mathbb{N} \to \mathbb{N} \times \mathbb{N}$.
 
\[
{<, >}^{- 1}(z) = (l(z), r(z))
\]
\noindent
En donde $l$ y $r$ son primitivas recursivas:
\begin{enumerate}
	\item[] $l(z) = max \; \{ k \in \mathbb{N} : 2^k \mid z + 1 \} = max_{k \leq z} \; 2^k \mid z + 1 = min_{k \leq z} \; \exists_{t \leq z} \; z = <k, t>$
	\item[] $r(z) = max \; \{ k \in \mathbb{N} : 2^k \mid z + 1 \} = max_{k \leq z} \; 2^k \mid z + 1 = min_{k \leq z} \; \exists_{t \leq z} \; z = <t, k>$
\end{enumerate}
 
\end{observation}
\noindent
\textbf{N\'umero de Godel de una Secuencia.} Si $(a_1, a_2, \ldots, a_n)$ es un vector de n-coordenadas en $\mathbb{N}^n$ se define \textbf{el n\'umero de Godel de la sucesi\'on }$a_1, a_2, \ldots, a_n$ al n\'umero $[a_1, a_2, \ldots, a_n] = \prime_{1}^{a_1}, \prime_{2}^{a_2}, \ldots, \prime_{n}^{a_n}$ en donde $\prime_1, \prime_2, \ldots, \prime_n$ es la sucesi\'on de los primeros n-n\'umeros primos
 
\begin{example}
\[
[3, 0, 1, 0, 4, 0, 5] = 2^3 \cdot 3^0 \cdot 5^1 \cdot 7^0 \cdot 11^4 \cdot 13^0 \cdot 17^5
\]
Para cada $n$, $[\, , \, , \, \ldots , \,] \colon \mathbb{N}^n \to \mathbb{N}$ es recursiva primitiva.
\end{example}
 
\begin{observation}
La funci\'on par $<x, y> = 2^{x} \cdot (2 \cdot y + 1) - 1$ deja de ser biyectiva si la entrada $(x, y) \in \mathbb{R}^2$
\end{observation}
 
\subsection{Codificaci\'on de Programas}
 
\begin{enumerate}
	\item Codificaci\'on de las instrucciones
		\begin{enumerate}
			\item \textbf{Asignar un n\'umero a cada variable}. Ordenamos las variables del 					siguiente modo $Y$, $X_1$, $Z_1$, $X_2$, $Z_2$, $\ldots$, $X_n$, $Z_n$, $\ldots$ si 				$V$ es una variable, $\#V$ es la ubicaci\'on de $V$ en esta lista.
			\item \textbf{Asignar un n\'umero a cada etiqueta}. Si $L$ es una etiqueta $\#L$ es la 			ubicaci\'on de $L$ en la siguiente lista: $A_1, B_1, C_1, D_1, A_2, B_2, C_2, D_2, 					E_2, \ldots$
		\end{enumerate}
		\noindent
		\textbf{Convenci\'on}. Si $V$ es una variable agregamos a la lista de instrucciones la 				macro $V \leftarrow V$.
	\begin{enumerate}
		\item Si $I$ es una instrucci\'on, definimos $\# I = <\underbrace{a}_{\text{etiqueta}}, < 			\overbrace{b}^{\text{tipo de instrucci\'on}}, \underbrace{c}_{\text{variable}}>>$.
		\begin{itemize}
			\item Si $I$ no tiene etiqueta $a = 0$
			\item Si $I$ tiene etiqueta $M$, $a = \# M$
			\item Si $c = \# V - 1$ donde $V$ es la \'unica variable que aparece en I.
			\item Si $I$ es $V \leftarrow V$, $b = 0$
			\item Si $I$ es $V \leftarrow V + 1$, $b = 1$
			\item Si $I$ es $V \leftarrow V - 1$, $b = 2$
			\item Si $I$ es $IF \, V \neq 0 \, GOTO \, L$, $b = \#L + 2$
		\end{itemize}
		\item Si $I$ y $J$ son instrucciones tal que $\# I = \#J$
			\begin{enumerate}
				\item [] $\# I = <a, <b, c>>$
				\item [] $\# J = <a', <b', c'>>$
			\end{enumerate}
			\noindent
			$\Rightarrow$ $a = a'$, $b = b'$ $c = c'$, por lo tanto la funci\'on par es biyectiva. 			Cada n\'umero natural es la codificaci\'on de una instrucci\'on.
	\end{enumerate}
	\begin{example}
		Encontrar la instrucci\'on $I / \# I = 40$
		
		\[
			<\underbrace{a}_{x}, \overbrace{<b, c>}^{y}>		
		\]
		
		\begin{equation}\label{E:x}
			x = max_{k} \; 2^k \mid 40 + 1 = 41 \Rightarrow x = 0 = a
		\end{equation}
					
		\begin{align*}
		 2y + 1 = 41		\\
		 2y		= 40		\\
		 y		= 20		\\
		 <b, c>	= 20
		\end{align*}
		
		Por el mismo razonamiento que~\eqref{E:x}, $b = 0$ ($I$ es $V \leftarrow V$)
		
		\begin{align*}
		 2y + 1 &= 21		&		 &													\\
		 2y		&= 20		&		 &													\\
		 y		&= 10		& c &= 10 & Y, X_1, Z_1, X_2, Z_2, X_3, Z_3, X_4, Z_4, X_5
		\end{align*}
		
		Entonces $I$ es $X_5 \leftarrow X_5$
		
	\end{example}
	\item Codificaci\'on de programas. Sea $P$ un programa y sean $I_1$, $I_2$, $\ldots$ $I_n$ las instrucciones de $P$. Definimos:
	
	\[
		\#\,P = [\#\,I_1, \#\,I_2, \#\,I_3, \ldots, \#\,I_n] - 1
	\]
	\noindent
	\begin{observation}
	$\#\,I = 0 \iff I$ es $Y \leftarrow Y$. Teniendo en cuenta esto, tomamos como \textbf{convenci\'on}: si $n > 1$, la instrucci\'on $I_n$ no puede ser la instrucci\'on $Y \leftarrow Y$ (para que haya unicidad).
	
	Usando la unicidad de factorizaci\'on en primos se deduce que la aplicacion $P \mapsto P$ es biyectiva.
	\end{observation}
	
	\begin{example}
	$\# \, P = 40 \iff [\#\,I_1, \#\,I_2, \#\,I_3, \ldots, \#\,I_n] = 41 = 2^0 \cdot 3^0 \cdot 5^0 \ldots \cdot 41^1$
	\end{example}
\end{enumerate}
 
\subsection{Aplicacion: Problema de la parada}
\hfill
\[
HALT(x, y) =
\begin{cases}
1				&\text{si el programa de codigo $y$ y entrada $x$ se detiene} \\
0				&\text{si no}												\\
\end{cases}
\]
 
\begin{theorem}[HALT no es computable]
HALT es un predicado que no es computable en el lenguaje $S$
\end{theorem}
 
\begin{proof}[Demostraci\'on]
Supongamos por el absurdo que el programa \textbf{HALT} es predicado computable. 

En particular $HALT(x, x)$ tambi\'en es computable. En particular el siguiente programa est\'a bien definido:
 
$[A]\;IF\;HALT(x,\;x)\;=\;1\;GOTO\;A$
 
Este programa $\mathcal{P}$ computa la siguiente funci\'on:
 
\[
\mathcal{F}(x) =
\begin{cases}
\uparrow		&\text{si $HALT(x, x) = 1$}		\\
0				&\text{si $HALT(x, x) = 0$}		\\
\end{cases}
\]
 
Sea $Z_0 = \#\mathcal{P}$ Cu\'anto vale $F(Z_0)$?
 
Si $\mathcal{F}(Z_0) \; \uparrow$ entonces $HALT(Z_0,Z_0) = 1$. Luego el programa se detiene con entrada $Z_0$ y luego $\mathcal{F}(Z_0)$ est\'a definida, lo que es imposible.
 
Si $\mathcal{F}(Z_0)$ entonces $HALT(Z_0, Z_0) = 0$ y luego $\mathcal{P}$ no se detiene con la entrada $Z_0$ y luego $\mathcal{F}(Z_0) \uparrow$ es absurdo
 
\end{proof}
 
\begin{definition}[Resultado de Cantor]
Si X es un conjunto, no existe ninguna biyecci\'on en $X$ y $\mathcal{P}(X)$ \{conjunto de partes de X \}. Sino existir\'ia $\mathcal{F} \colon (X) \rightarrow P(X)$.
\end{definition}
 
\begin{proof}[Demostraci\'on]
Sea $A = { x \in X | x \notin F(x) }$. Sea $Z_0 \in X | \mathcal{F}(Z_0) = A$. $Z_0 \in A \iff Z_0 \in \mathcal{F}(Z_0)$ y $Z_0 \in A \iff Z_0 \notin \mathcal{F(Z_0)}$ absurdo
\end{proof}
 
