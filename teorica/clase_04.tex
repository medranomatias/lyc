\section{Clase 04}

\subsection{Alfabeto Com\'un de la L\'ogica de Primer Orden}

\begin{enumerate}
	\item Variables: $\{x_0, x_1, \ldots, x_n\}$
	\item Conectivos: $\{\land, \lor, \Rightarrow, \neg, \forall, \exists \}$
	\item Par\'entesis: $\{ (, ) \}$
\end{enumerate}

El alfabeto de un \textit{lenguaje de primer orden} contiene adem\'as de los s\'imbolos anteriores, el siguiente conjunto de s\'imbolos:

\begin{enumerate}
	\item Un conjunto $\mathcal{C}$ de s\'imbolos (eventualmente vac\'io) cuyos elementos se denominan 				\textit{s\'imbolos de constante}
	\item Un conjunto $\mathcal{F}$ (eventualmente vac\'io) cuyos elementos se denominan \textit{s\'im\-bolos de funcion}
	
	M\'as precisamente, un s\'imbolo de funci\'on n-ario es un s\'imbolo que se representa como $f^n$ y $n \in \mathbb{N}, n \geq 1$. Notaremos con $F^n$ al conjunto de s\'imbolos n-arios. Luego $\mathcal{F} = \cup_{n \geq 1} \mathcal{F}^n$.
	\item Un conjunto no vac\'io de $\mathcal{P}$ cuyos elementos se denominan \textit{s\'imbolos de relaci\'on}. M\'as precisamente, un s\'imbolo de relaci\'on n-ario es un s\'imbolo que representa como $p^n$ siendo $n \geq 1$.
	
	Notaremos con $\mathcal{P}^n$ al conjunto de los s\'imbolos de relaci\'on n-ario. Luego $\mathcal{P} = \cup_{n \geq 1} \mathcal{P}^n$
\end{enumerate}

A partir de un alfabeto dado por los s\'imbolos b\'asicos, de constante, de funci\'on y de relaci\'on, podemos definir la noci\'on de lenguaje de primer orden.

Un lenguaje de primer orden consiste en dos tipos de expresiones: \textbf{t\'erminos} y \textbf{f\'ormulas}.

\begin{definition}

Un \textit{t\'ermino} es una expresi\'on que se define inductivamente como sigue:

\begin{enumerate}
	\item Un s\'imbolo de constante es t\'ermino.
	\item Las variables son t\'erminos.
	\item Si $\mathcal{F}^n$ es un s\'imbolo de funci\'on n-ario y $t_1, t_2, \ldots, t_n$ t\'erminos, entonces
	$F^n \; t_1, t_2, \ldots, t_n$ es un t\'ermino.
	\item Una expresi\'on $t$ es un t\'ermino si y s\'olo si se obtiene de un n\'umero finito de pasos usando las reglas anteriores.
\end{enumerate}

\end{definition}

Una definici\'on alternativa: una cadena de formaci\'on de t\'erminos es una sucesi\'on finita de $t_1, t_2, \ldots, t_n$ tal que si $1 \leq i \leq n$ $t_i$ es una variable $x_j$ o $t \in \mathcal{C}$ o $t \in \mathcal{F}^k \; t_1, t_2, \ldots, t_k$ con $\mathcal{F}^k$ s\'imbolo de funci\'on k-ario y $t_1. t_2, \ldots, t_k$ t\'erminos.

\begin{example}

Un ejemplo proveniente del \'algebra. Sea $P(x) = 3x^5 + x^4 + x - 3$ aqu\'i las funciones involucradas son + y . (funciones binarias) y las constantes 3, 1 y -3.

Otro ejemplo, sea A alfabeto que contiene un \'unico s\'imbolo de funci\'on unaria $F^1$. Ejemplos de terminos: $x_1, F'x_1, \underbrace{F'F'F'F' \ldots F'x_1}_{k veces}$

\end{example}

\begin{definition}

Una \textbf{f\'ormula at\'omica} es una expresi\'on del tipo $\alpha = \mathcal{P}^n(t_1, t_2, \ldots, t_n)$ donde $\mathcal{P}^n$ es un s\'imbolo de relaci\'on n-ario y $t_1, t_2, \ldots, t_n$ son t\'erminos

\end{definition}

\begin{example}

$x_1 = x_2$

$\exists x_1, \exists x_2, x_1 = x_2$


\end{example}

\begin{definition}

Una \textbf{cadena de formaci\'on de f\'ormulas} es una sucesi\'on finita de $c_1, c_2, \ldots, c_n$ de expresiones sobre un alfabeto de una l\'ogica de formaci\'on de $n$ tal que si $1 \leq i \leq n$, entonces $c_i$ es una f\'ormula at\'omica o $\exists j < i / c_i = \neg c_j$ o $\exists j, k < i$ y $* \in \{\land, lor,  \Rightarrow \}$ tal que $c_i = (c_j * c_k)$ o existe una variable $x_k$ y $j < i / c_i = \forall x_k c_j$ o $c_i = \exists x_k c_j$

\end{definition}

\begin{definition}

Una expresi\'on $\alpha$ se dice \textbf{formula} si existe una cadena de formaci\'on (de f\'ormulas) $c_1, c_2, \ldots, c_n$ tal que $c_n = \alpha$.

\end{definition}

\begin{example}

Sea $\mathcal{P}^1$ un s\'imbolo de relaci\'on unario y supongamos que el alfabeto no contiene s\'imbolos de funci\'on ni de constante.

En este alfabeto las f\'ormulas at\'omicas son de tipo $\mathcal{P}^1(x_1)$ con $x_1$ variables.

Ejemplo:

$\neg P^1(x_j)$

$\exists x_1 \neg P^1(x_2)$ (Nota: Desde el punto de vista formal, est\'a permitido)

$\forall x_2 \; \exists x_2 P^1(x_1)$

\end{example}

Un lenguaje de \textbf{Primer Orden} construido a partir de un alfabeto A que contiene a las variables, a los par\'entesis y a los conectivos $\{ \neg, \lor, \land, \Rightarrow, \forall, \exists \}$. Es un lenguaje cuyas expresiones son \textbf{t\'erminos} o \textbf{f\'ormulas}.

\begin{observation}
$\;$
\begin{enumerate}
	\item El alfabeto A puede ser infinito.
	\item El t\'ermino general \textit{Primer Orden} proviene del hecho que solo se pueden cuantificar variables.
\end{enumerate}

\end{observation}

\begin{example} Consideremos el siguiente alfabeto
	
	$\{F^1\} \cup \{ variables \} \cup \{ conectivos\} \cup \{ (, )\}$
	
	$\forall x_1 P^1 (F^1 x_1)$ \textit{es formula}.
	
	$\forall F^1 P^1 (F^1 x_1)$ \textbf{no es f\'ormula}. No se pueden cuantificar s\'imbolos, 			solamente se pueden cuantificar variables.		 
	\item \textit{lenguaje de primer orden} se refiere a que cada lenguaje de primer orden 			depende de los s\'imbolos \textit{extras} que aparecen en los diferentes alfabetos 		

\end{example}

\subsection{Unicidad de la Escritura}

\begin{theorem}
$\;$
\begin{enumerate}
	\item Si $\mathfrak{L}$ es un lenguaje de primer orden y $t$ es un t\'ermino se da una y solo una de las siguientes condiciones.
	\begin{enumerate}
		\item $t$ es una variable $x_1$
		\item $t$ es un s\'imbolo de constante
		\item Existe un \'unico s\'imbolo de funci\'on $f^k$, \textit{k-ario} y \'unicos t\'erminos $t_1, t_2, \ldots, t_k$ tal que $t = f^k t_1, t_2, \ldots, t_k$		
	\end{enumerate}
	\item Si $\alpha$ es una f\'ormula entonces se verifica una y solo una de las siguientes condiciones.
	\begin{enumerate}
		\item $\alpha$ es una f\'ormula at\'omica
		\item $\alpha = \neg \beta$ con $\beta$ f\'ormula.
		\item $\alpha = (\alpha_1 * \alpha_2)$ con $\alpha_1, \alpha_2$ formulas y $* \in \{ 				\land, \lor, \rightarrow \}$
		\item O existe una \'unica variable $x_1 \ \alpha = \forall x_j \beta$ o existe una \'unica 			variable $x_j \ \alpha = \exists x_j \beta$ siendo $\beta$ \'unica.
		
		La prueba se hace en ambos casos con inducci\'on completa.

		Para la prueba de los t\'erminos se usa inducci\'on en la complejidad de un t\'ermino t:

		$c(t)$ = "n\'umeros de s\'imbolos funcionales que aparecen en t".

		En este caso $c(t) = 0$ $\iff$ t es una variable o t es un s\'imbolo constante
		
		$\alpha = \forall x_j, \beta = \forall x'_j B'$ de este modo verifica la unicidad de la escritura.
		
		
	\end{enumerate}
\end{enumerate}

\end{theorem}