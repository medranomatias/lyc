\section{Clase 09}

\subsection{Notas}

\begin{enumerate}
	\item Modus Ponens: $\{P, \; (P \rightarrow Q) \} \vdash Q$
	\item Generalizaci\'on: $Q \vdash \forall x_i Q$
\end{enumerate}

$\gamma_1, \gamma_2, \ldots, \gamma_n$ si $1 \leq i \leq n$ en donde $\gamma_i$ es un axioma, si se obtiene a partir de eslabones anteriores por 1 o 2.

Sea $\mathcal{L} = \{ \mathcal{P}^2 \}$, sea $\mathcal{T}$ la teor\'ia de relaciones de equivalencia:

\begin{enumerate}
	\item Axioma 1: $\forall x_1 \; \mathcal{P}^2(x_1, x_1)$
	\item Axioma 2: $\forall x_1, \forall x_2 \; ( \mathcal{P}^2(x_1, x_2) \rightarrow \mathcal{P}^2(x_1, x_2))$
	\item Axioma 3: $\forall x_1, \forall x_2, \forall x_3 \; (\mathcal{P}^2(x_1, x_2) \land \mathcal{P}^2(x_2, x_3) \rightarrow \mathcal{P}^2(x_1, x_3))$
\end{enumerate}

\begin{definition}
Sea $\mathcal{L}$ un lenguaje de primer orden y sea $\Gamma$ un conjunto de f\'ormulas de $\mathcal{L}$ . Si $\alpha$ es una f\'ormula de $\mathcal{L}$ diremos que $\alpha$ es \textbf{deducible} de $\Gamma$ y se nota $\Gamma \vdash \alpha$, si existe una secuencia de f\'ormulas de $\mathcal{L} \colon \gamma_1, \gamma_2, \ldots, \gamma_n$ tal que $\gamma_n = \alpha$ y $\forall 1 \leq i \leq n$ se verifica una y solamente una de las siguientes condiciones:

\begin{enumerate}
	\item $\gamma_i$ es un axioma l\'ogico.
	\item $\gamma_i \in \Gamma$ 
	\item $\gamma_i$ se obtiene por modus ponens.
	\item $\gamma_i$ se obtienen por generalizaci\'on.
\end{enumerate}
\end{definition}

\begin{definition}

Diremos que $\alpha$ es \textbf{demostrable a partir de $\Gamma$} si existe una formula $\gamma_1, \gamma_2, \ldots, \gamma_n \; / \; \gamma_n = \alpha$ y $\gamma_1, \ldots, \gamma_n$ verifica la definici\'on anterior.

\end{definition}

\begin{observation}

Si $\gamma = \emptyset$, $\emptyset \vdash \lambda$ si y s\'olo si $\alpha$ es demostrable

\end{observation}

\begin{example}

$\mathcal{L} = \{ \mathcal{P}^2\}$ y $\gamma$ son los axiomas 1, 2 y 3 de la teor\'ia de las relaciones de equivalencia.

Sea $\lambda$ la siguiente f\'ormula:

$\forall x_1, \forall x_2 (\exists x_3 (\mathcal{P}^2(x_1, x_3), \mathcal{P}^2(x_2, x_3)))$ 
$\rightarrow$ $\forall x_4 \mathcal{P}^2(x_1, x_4) \iff \mathcal{P}^2(x_2, x_4)$

$(\lambda \iff \beta) (\lambda \rightarrow \beta) \land (\beta \rightarrow \lambda)$

$\gamma \vdash \alpha$ (En este ejemplo $\alpha$ es enunciado y $\gamma$ es teor\'ia)

\end{example}

\begin{definition}

Si $\mathcal{L}$ es un lenguaje de primer orden y $\Gamma$ un conjunto de f\'ormulas de $\alpha$ y $\alpha$ es una f\'ormula de $\mathcal{L}$, diremos que $\alpha$ es \textbf{consecuencia l\'ogica} de $\Gamma$, y escribimos $\Gamma \models \alpha$ si se verifica la siguiente condici\'on:

Si $\mathcal{I}$ es una interpretaci\'on de $\mathcal{L}$ con dominio $\mathcal{D}$ entonces si $S$ es una sucesi\'on de elementos de $\mathcal{D}$ que satisface a todas las f\'ormulas de $\Gamma$ entonces $S$ debe satisfacer a $\alpha$.

Caso particular $\Gamma$ es una teor\'ia y $\alpha$ es un enunciado. Luego $\Gamma \models \gamma$ si y s\'olo si cada vez que $\mathcal{I}$ es un modelo de $\Gamma$ entonces $\mathcal{I}$ debe ser modelo de $\alpha$

\end{definition}

\begin{example}

Si $\Gamma = \emptyset$ $\emptyset \models \alpha \iff \alpha$ es l\'ogicamente v\'alido.

\end{example}

\begin{proposition}

Si $\mathcal{L}$ es un lenguaje de primer orden, $\Gamma$ un conjunto de f\'ormulas de $\mathcal{L}$ y $\alpha$ una f\'ormula. Si $\Gamma \vdash \alpha$ entonces $\Gamma \models \alpha$
 
En particular, si $\Gamma = \emptyset$, $\alpha$ demostrable implica $\alpha$ es l\'ogicamente v\'alida.

Prueba: Es an\'aloga a la prueba del \textbf{Teorema} que prueba que toda f\'ormula es demostrable es l\'ogicamente v\'alida

\end{proposition}

\subsection{Teorema de la Deducci\'on}

Sea $\Gamma$ un conjunto de f\'ormulas y sea $\alpha, \beta$ f\'ormulas de un lenguaje $\mathcal{L}$ tales que $\Gamma \cup \{\alpha \} \vdash \beta$. Entonces $\Gamma \vdash (\alpha \models \beta)$ si en la deducci\'on de $\beta$ a partir de $\Gamma \cup \{ \alpha \}$ ninguna aplicaci\'on de la regla de la generalizaci\'on contenga una variable $x_k$ que aparezca libre en $\alpha$

\begin{proof}

Sea $\gamma_1, \gamma_2, \ldots, \gamma_n$ una prueba de $\beta$ a partir de $\Gamma \cup \{ \alpha\}$ Probemos por inducci\'on que $\Gamma \vdash (\alpha \rightarrow \gamma_i)$ $\forall 1 \leq i \leq n$

Si $i = 1$ $\gamma_1$ es un axioma o $\gamma_1 \in \Gamma$ o $\gamma_1 = \alpha$

Si $\gamma_1$ es axioma, construimos la siguiente prueba de $\alpha \rightarrow \gamma_1$

$\gamma_1$

$(\gamma_1 \rightarrow (\alpha \rightarrow \gamma_1))$

(M.P.) $\alpha \rightarrow \gamma_1$

Si $\gamma_1 \in \Gamma$ es la misma prueba. 

Si $\gamma_1 = \alpha$ ya vimos que $\alpha \rightarrow \alpha$ es demostrable y luego $\Gamma \models (\alpha_i \rightarrow \alpha_i)$

%Paso Inductivo
Supongamos que $1 < j \leq n$ y veamos qu\'e $\Gamma \vdash (\alpha \rightarrow \gamma_j)$

Si $\gamma_j$ es axioma o $\gamma_j = \alpha$ o $\gamma_j \in \Gamma$ es la misma prueba que el caso base. Sino hay dos subpruebas:

\begin{enumerate}
	\item Si $\gamma_j$ se obtiene por modus ponens: es decir $\exists i, k < j \; \/ \gamma_k = (\gamma_i \rightarrow \gamma_j)$ $\gamma_i$ $\gamma_i \rightarrow \gamma_j \ldots \gamma_j$
	
	Por hip\'otesis inductiva $\Gamma \vdash (\alpha \rightarrow \gamma_i)$ y $\Gamma \vdash (\alpha \rightarrow (\gamma_i \rightarrow \gamma_j))$
	
	$((\alpha \rightarrow (\gamma_i \rightarrow \gamma_j)) \rightarrow ( (\alpha \rightarrow \gamma_i) \rightarrow (\alpha \rightarrow \gamma_j)))$ 
	
	$(\alpha \rightarrow (\gamma_i \rightarrow \gamma_j))$
	
	(M.P.) $((\alpha \rightarrow \gamma_i) \rightarrow (\alpha \rightarrow \gamma_j))$
	
	$(\alpha \rightarrow \gamma_i)$
	
	M.P. $\alpha \rightarrow \gamma_j$
	
	\item $\gamma_j$ se obtienen por generalizaci\'on $\gamma_j = \forall x_k \gamma i$ con $i < j$ y $x_k$ no es libre en $\alpha$.
	
	Por H.I. $\Gamma \vdash (\alpha \rightarrow \gamma_j)$, usamos el siguiente axioma
	
	$(\forall x_k (\alpha \rightarrow \gamma_i) \rightarrow (\alpha \rightarrow \underbrace{\forall x_k \; \gamma_i}_{\gamma_j}))$
		
\end{enumerate}

Consideremos la siguiente prueba de $(\alpha \rightarrow \gamma_j)$ a partir de $\Gamma$

\begin{enumerate}
	\item $\alpha \rightarrow \gamma_i$
\end{enumerate}

\end{proof}

Ejemplo:

$x_i = c$ aplicar la regla de generalizaci\'on y queda como $\forall x_i \; x_i = c$

$\underbrace{x_i = c}_{\alpha} \rightarrow \underbrace{\forall x_i (x_i = c)}_{\beta})$ es l\'ogicamente v\'alida.

Sea $\mathcal{I} = (\{1, 2\}, \underbrace{1}_{\text{constante}})$. Con esta interpretaci\'on, falla al momento de asignar el valor $i = 2$

\subsection{Corolario}

Si $\alpha$ es un enunciado y $\Gamma$ es un conjunto de f\'ormulas entonces para toda f\'ormula $\beta$ si $\Gamma \cup \{ \alpha \} \vdash \beta$ entonces $\Gamma \vdash (\alpha \rightarrow \beta)$

Caso particular: $\Gamma$ un conjunto finito de enunciados $\Gamma = \{ \gamma_1, \gamma_2, \ldots, \gamma_n\}$ $\{ \gamma_1, \gamma_2, \ldots, \gamma_n\} \vdash \beta$ implica $\emptyset \vdash (\gamma_1 \rightarrow (\gamma_2 \rightarrow \ldots (\gamma_n \rightarrow \beta)))$

%verificar

Ejemplo:

Probar que la siguiente f\'ormula es l\'ogicamente v\'alida

$\forall x_1, \forall x_2 \alpha \rightarrow \forall x_2, \forall x_1 \alpha$

Basta con ver que $\emptyset \vdash \forall x_1, \forall x_2 \alpha \rightarrow \forall x_2, \forall x_1 \alpha$

Por el teorema de la deducci\'on basta probar $\forall x_1, \forall x_2 \alpha \vdash \forall x_2, \forall x_1 \alpha$

Consideremos la siguiente prueba

\begin{enumerate}
	\item $\forall x_1, \forall x_2 \alpha$
	\item axioma 5 $\forall x_1 \varphi \rightarrow \varphi(x_i, t)$
	
	$\varphi = \forall x_2 \alpha$ y $t = x_i$
	
	$(\forall x_1, \forall x_2 \; \alpha \rightarrow \forall x_2 \alpha)$
	
	\item M.P. $\forall x_2 \; \alpha$
	
	\item AX5 $(\forall x_2 \alpha \rightarrow \alpha)$
	
	\item MP $\alpha$
	
	\item $\forall x_i \alpha$
	
	\item $\forall x_2, \forall x_1, \alpha$
\end{enumerate}



%inconsistencia

T se dice \textbf{inconsistente} si existe $\alpha$

$T \vdash \alpha$

$T \vdash \neg \alpha$

Proposici\'on: Si una teor\'ia T es inconsistente entonces para toda f\'ormula $\beta$, $T \vdash \beta$

Prueba: Consideremos el axioma 3

