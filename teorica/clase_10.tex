\section{Clase 10}

\subsection{Notas}
$\;$

$\Gamma \vdash \alpha \iff \{ \alpha \text{ es demostrable a partir de los axiomas logicos } \} \iff \emptyset \vdash \alpha$. Nocion sintactica

$\Gamma \models \alpha \iff \{ \alpha \text{ es consecuencia logica de } \Gamma \}$. Nocion semantica

\begin{observation}

Si $\Gamma \vdash \alpha$, existe una prueba $\gamma_1, \gamma_2, \ldots, \gamma_n$ de $\alpha$. Entonces existe $\Gamma_{'} \subseteq \Gamma$, $\Gamma_{'}$ finita / $\Gamma_{'} \vdash \alpha$.

\end{observation}

\begin{observation}

Sea $\mathcal{L}$ un lenguaje de primer orden y sea $\mathcal{I}$ una interpretacion del lenguaje $\mathcal{L}$.

Sea $\mathcal{F}$ el conjunto de todas las formulas de $\mathcal{L}$. Sea $\mathcal{F}_{I} = \{ \alpha \in \mathcal{F} / \alpha \text{ es valida en } \mathcal{I} \}$ 

($\alpha$ es valida en $\mathcal{I}$ si toda sucesion $S$ con valores en el dominio de $\mathcal{I}$ satisface a $\alpha$) 

$\iff$ $\forall x_1, \forall x_2, \ldots, \forall x_n $ $\alpha$ es un enunciado valido en $\mathcal{I}$ con $x_1, x_2, \ldots, x_n$ las variables libres de $\alpha$

\end{observation}

\begin{proposition}

El conjunto de $\mathcal{F}_{I}$ verifica las siguientes condiciones:

\begin{enumerate}
	\item Si $\varphi$ es un enunciado de $\mathcal{L}$ entonces $\varphi \in \mathcal{F}_{I}$ \'o $\neg \varphi \in \mathcal{F}_{I}$
	\item El conjunto $\mathcal{F}_{I}$ no puede tener ninguna contradiccion. ($\nexists \alpha $ formulas ($\alpha \in \mathcal{F}$) tal que $\alpha \in \mathcal{F}_{I}$ y $\neg \alpha \in \mathcal{F}_{I}$)
	\item $\mathcal{F}_{I}$ es cerrado por las reglas de inferencia, es decir:
		\begin{enumerate}
			\item Si $\alpha \in \mathcal{F}_{I}$ y $(\alpha \rightarrow \beta) \in \mathcal{F}_{I}$ entonces $\beta \in \mathcal{F}_{I}$
			\item Si $\alpha \in \mathcal{F}_{I}$ y $x_j$ una variable entonces $\forall x_j \; \alpha \in \mathcal{F}_{I}$
		\end{enumerate}
	\item $\mathcal{F}_{I}$ contiene a los axiomas logicos (3, 4 ligadas a lo sintactico)
\end{enumerate}

\end{proposition}

\begin{definition}

	Sea $\mathcal{L}$ un lenguaje de primer orden y $\Gamma$ un conjunto de enunciados $\Gamma$ se dice \textbf{completo} si se verifica la condicion 1 de la proposicion anterior.
		
\end{definition}

\begin{definition}

Diremos que $\Gamma$ es un conjunto de \textbf{Goedel} si se verifican las 4 
condiciones de la proposicion anterior.

\end{definition}

\begin{observation}

El conjunto $\mathcal{F}_{I}$ contiene en particular a las tautologias

\begin{center}
\begin{tabular}{p{0.4\textwidth}p{0.4\textwidth}}
  \hline			
  P & $P \rightarrow P$ \\ \hline
  1 & 1  \\ \hline
  0 & 1  \\ \hline  
\end{tabular}
\end{center}

\end{observation}

\textquestiondown Como definir tautologia en logica de primer orden?

Fijemos un lenguaje $\mathcal{L}$ y un conjunto $S$ de formulas. A partir del conjunto $S$ podemos asociar una cadena de formacion. Dicha cadena de formacion de una secuencia $\gamma_1, \gamma_2, \ldots, \gamma_n$ tal que $\forall 1 \leq i \leq n$, $\gamma_i \in S$ \'o $\gamma_i = \neg \gamma_j $ con $j < i$ \'o $\gamma_i = (\gamma_j \bullet \gamma_k)$ con $j, k < i$, $\bullet \in \{ \land, \lor, \rightarrow\}$

Si fijamos una funcion $v \colon S \rightarrow \{0, 1\}$. En donde $v$ indica un valor de verdad en cualquier formula $\gamma$ / existe una cadena de formacion $\gamma_1, \ldots, \gamma_n$ y $\gamma_n = \gamma$

$v(\gamma_1)$ es dato pues $\gamma_1 \in S$

Si $i > 1$ e $\gamma_i = \neg \gamma_j$ con $j < i$ $v(\gamma_i) = 1 - v(\gamma_j)$

Si $\gamma_i = (\gamma_j \lor \gamma_k)$ con $j, k < i$ $v(\gamma_i) = max \{ v(\gamma_j), v(\gamma_k) \}$

Si $\gamma_i = (\gamma_j \land \gamma_k)$ con $j, k < i$ $v(\gamma_i) = min \{ v(\gamma_j), v(\gamma_k) \}$

Si $\gamma_i = (\gamma_j \rightarrow \gamma_k)$ con $j, k < i$ $v(\gamma_i) = max \{1 - v(\gamma_j), v(\gamma_k) \}$

\begin{definition}

Si $\mathcal{L}$  es un lenguaje de primer orden y $\gamma$ es una formula, $\gamma$ se dice \textbf{tautologia} si existe un $S \subseteq \mathcal{F}$ y una cadena de formacion $\gamma_1, \gamma_2, \ldots, \gamma_n$ de $\gamma$ a partir de $S$ tal que $v(\gamma) = 1$ para cualquier asignacion $v \colon S \rightarrow \{0, 1 \}$

\begin{example}
$(P \rightarrow (Q \rightarrow P))$ es tautologia. En donde $S = \{ P, Q\}$. El conjunto de $\mathcal{F}_{I}$ contiene una tautologia.
\end{example}

\end{definition}

