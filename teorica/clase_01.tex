\section{Clase 01}

\begin{definition}
Sea conjunto X y \relacion una relaci\'on binaria de a \relacion b. \relacion es 
de \textbf{equivalencia} si se verifica que:

\begin{enumerate}
\item reflexiva \cforall x, x \relacion x
\item simetrica \cforall x, y, x \relacion y \then y \relacion x
\item transitiva \cforall x, y, z, x \relacion y y \relacion z \then x
\relacion z
\end{enumerate}

\end{definition}

\begin{theorem}
Si dos elementos son equivalentes a un tercero, entonces tiene los mismos 
elementos equivalentes:
\hfill \break
\cforall x, y, \cexists u, x \relacion u \cland y \relacion u \then \cforall z
(x \relacion z \ciff y \relacion z)
\end{theorem}

\begin{proof}
Como y \relacion u entonces de 2*  se obtiene u \relacion y, como x \relacion u
y u \relacion y por 3* se obtiene x \relacion y. Nuevamente, por 2* se deduce
que tambien y \relacion x.
\hfill \break
Sea z arbitrario, si x \relacion z, cada y \relacion x por 3* se obtiene y
\relacion z. Si y \relacion z, como x \relacion y entonces x \relacion z. 
\end{proof}

\subsection{Convenci\'on} 

Notaremos \cN al conjunto de naturales  
 
\subsection{Principio de inducci\'on completa}
 
Sea P(n) una proposici\'on sobre \cN tal que: 

\begin{enumerate}
  \item P(0) es verdadera
  \item Si P(k) es verdadero \cforall k \< n \then P(n) es verdadera  
\end{enumerate} 

\subsection{Concepto de Lenguajes Formales}
 
Sea A un conjunto no vac\'io llamado alfabeto: 

\begin{definition}
Una expresi\'on sobre A es una sucesi\'on finita de elementos de A. Notaremos con A\supi{*} al conjunto de todas las expresiones sobre A. Es
decir E \cin A\supi{*} al conjunto de todas las expresiones sobre
A. Es decir E \cin A\supi{*} \ciff \cexists n \cin
\cN y elementos a\subi{0}, a\subi{1}, \ldots, a\subi{n}
tal que E = a\subi{0}, a\subi{1}, \ldots, a\subi{n}.
\end{definition}

\begin{definition}
Si E \cin A\supi{*} se define a \textbf{longitud} de E al n\'umero \length(E) = n +
1
\end{definition}

\begin{definition}
Si A es un conjunto no vacio y E,F \cin A\supi{*} se define E.F la
expresi\'on siguiente. Si E = a\subi{0}, a\subi{1}, \ldots,
a\subi{n} y F = b\subi{0}, b\subi{1}, \ldots , b\subi{n}

E.F = a\subi{0}, a\subi{1}, \ldots, a\subi{n}, b\subi{0},
b\subi{1}, \ldots , b\subi{n} 

E.F se denomina la \textbf{concatenaci\'on} de E con F. 

\end{definition}

\begin{observation}
\length(EF) = \length(E) + \length(F). En general, E.F \cneq F.G.
\end{observation}

\begin{definition}
Si A es un conjunto no vac\'io y E,F  \cin A\supi{*}. Diremos que F
es una secci\'on inicial de E si \cexists G tal que F.G = E
\end{definition}

\begin{example}
Si A es el alfabeto de la lengua espa\~nola. E = CASAMIENTO, F = CASA 
\end{example}

\begin{proposition}
Si A es un alfabeto y E,F,G y H \cin A\supi{*}, verifican E.F =
G.H, entonces E es secci\'on inicial de G o G es secci\'on inicial de E. M\'as a\'un, si
\length(E) = \length(G) entonces E = G. Si \length(E) \< \length(G), entonces E es una secci\'on inicial
de G y si \length(E)\(>\) \length(G) entonces G es una secci\'on inicial de E. 
\end{proposition}

\begin{proof}
Supongamos primero que \length(E) \(>\) \length(G) . Usaremos el principio de inducci\'on.  

Sea P(n): Si E,F,G,H son expresiones tales que \length(E) = n, n \(>\) \length(G) y E.F = G.H entonces G es secci\'on inicial de E. (nota: las expresiones son iguales cuando tienen la misma longitud y misma expresi\'on). 

Caso Base: P(0) Como el antecedente es falso (0 \(>\) \length(G)) por lo tanto la expresi\'on es verdadera.

Paso Inductivo: Supongamos que, por hip\'otesis inductiva, que P(n) es verdadera y probemos que P(n + 1). Sean E,F,G,H \cin A\supi{*} tales que \length(E) \(>\) \length(G) y E.F = G.H. E = a\subi{0}, a\subi{1}, \ldots, a\subi{n}, luego (a\subi{0}, a\subi{1}, \ldots, a\subi{n}).F = (a\subi{0},b\subi{1}, \ldots, b\subi{k}).H y k \< n luego por simplificaci\'on (a\subi{1}, \ldots, a\subi{n}).F = (b\subi{1}, \ldots, b\subi{k}).H. Aqu\'i vemos que tenemos dos expresiones completamente nuevas, deja de ser E y G. Las llamamos E' y G':

E' = (a\subi{1}, \ldots, a\subi{n})

G' = (b\subi{1}, \ldots, b\subi{k}) 

Entonces nos queda como E'F = G'H y \length(E') = n \(>\) \length(G') = k. Por hip\'otesis
inductiva, vale P(n), lo que implica que G' es secci\'on inicial de E' y luego E' = G'K  con K \cin A\supi{*}.

Entonces a\subi{0} E' = a\subi{o} G'K \then E = G.K

Si \length(E) = \length(G) \then E = G (ejercicio)
  
Si \length(E) \< \length(G)  la prueba es an\'aloga al primer caso.

\end{proof}

\begin{definition}
Sea A un alfabeto. Un lenguaje \(\lambda\) sobre A es un subconjunto no vac\'io de
A\supi{*}, es decir, \(\lambda\) \cneq \cemptyset y \(\lambda\)
\csubseteq A\supi{*}
\end{definition}

\begin{example}
\begin{enumerate}
  \item A es el alfabeto que consiste en letras \{a, \ldots, z\} y \(\lambda\) es el conjunto de palabras que aparecen en el diccionario de la Real Academia Espa\~nola.
  \item A es el alfabeto que consiste en los numeros \{0, \ldots, 9\} y \(\lambda\) =   \{ E \cin A\supi{*} / E representa un n\'umero natural \}
  \item E \csubseteq \(\lambda\) \ciff \length(E) = 1 \'o \length(E) $>$ 1 y el primer s\'imbolo de E debe ser diferente de 0.
  \item Construimos un alfabeto A (finito) y un lenguaje \(\lambda\) de modo tal que represente a los n\'umeros racionales. A = \{0, 1, 2, 3, \ldots ,  9\} \ccup \{,\} \ccup \{ \cland \}.
  \item Sea A = \{A, T, C, G \} \(\lambda\) = \{ E \cin A\supi{*} / E representa una cadena de genes \} (el lenguaje del ADN).
\end{enumerate} 
\end{example}

\subsection{El lenguaje de la l\'ogica proposicional}

A = \{ p\subi{0}, p\subi{1}, \ldots , p\subi{n}, \ldots \} \ccup \{ \cland, \clor, \then, \cneg \} \ccup \{ (, ) \}

Para cada i \cin \cN, p\subi{i} se denomina variable. Los elementos
\{ \cland, \clor, \then, \cneg \} se denominan conectivos y los elementos de tercer conjunto par\'entesis
\begin{center}
\begin{tabular}{ l | c }
  \hline			
	\clor	& \textit{disyunci\'on}		\\
	\cland	& \textit{conjunci\'on}		\\
	\then	& \textit{implicaci\'on}	\\
	\cneg	& \textit{negaci\'on}		\\
  \hline  
\end{tabular}
\end{center} 
El lenguaje asociado a este alfabeto se define de la siguiente modo. Notaci\'on, con $\mathcal{F}$ a dicho lenguaje. Los elementos de $\mathcal{F}$ se definen  del siguiente modo y se llamaran \textbf{f\'ormulas} de acuerdo a las siguientes reglas:

\begin{enumerate}
  \item Las variables p\subi{i} son f\'ormulas para \cforall i \cin \cN
  \item Si \(\alpha\) \cin $\mathcal{F}$ entonces \cneg \(\alpha\) \cin F. 
  \item Si \(\alpha\) y \(\beta\) \cin $\mathcal{F}$ entonces (\(\alpha\) \clor
  \(\beta\)) \cin $\mathcal{F}$ -asi con cada uno de los conectores-
  \item Una expresi\'on es formula si y s\'olo si se la obtiene en un n\'umero finito
  de pasos usando las reglas de 1, 2 y/o 3
\end{enumerate} 