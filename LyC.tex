\documentclass{amsart}

\usepackage[paper=a4paper, left=1.5cm, right=1.5cm, bottom=1.5cm, top=1.5cm]{geometry}

\usepackage{amssymb}
\usepackage{dsfont}
\usepackage{amsmath}
\usepackage{amssymb,latexsym}
\usepackage{graphicx}
\usepackage{tabto}
\usepackage{subfiles}

\newtheorem*{theorem}{Teorema}
\newtheorem*{lemma}{Lema}
\newtheorem*{definition}{Definici\'on}
\newtheorem*{observation}{Observaci\'on}
\newtheorem*{proposition}{Proposici\'on}
\newtheorem*{convencion}{Convenci\'on}
\newtheorem*{colorario}{Colorario}
\newtheorem*{aplicacion}{Aplicaci\'on}



\usepackage{symbols/symbol}
\usepackage{symbols/relation}
\usepackage{symbols/set}
\usepackage{symbols/logical}
\usepackage{symbols/other}

\renewcommand\qedsymbol{$\blacksquare$}

\renewcommand{\thefootnote}{\arabic{footnote}}

%sudo apt-get install xzdec
%tlmgr init-usertree
%tlmgr install xzdec
%tlmgr install fontspec
%\usepackage{fontspec}
\usepackage{mdframed} % Add easy frames to paragraphs
\usepackage{lipsum} % For dummy text
\usepackage{xcolor}
\usepackage{xparse} % Add support for \NewDocumentEnvironment
\definecolor{graylight}{cmyk}{.30,0,0,.67} % define color using xcolor syntax

\newmdenv[ % Define mdframe settings and store as leftrule
  linecolor=graylight,
  topline=false,
  bottomline=false,
  rightline=false,
  skipabove=\topsep,
  skipbelow=\topsep
]{leftrule}

\NewDocumentEnvironment{example}{O{\textbf{Ejemplo:}}} % Define example environment
{\begin{leftrule}\noindent\textcolor{graylight}{#1}\par}
{\end{leftrule}}

\NewDocumentEnvironment{exercise}{O{\textbf{Ejercicio:}}} % Define example environment
{\begin{leftrule}\noindent\textcolor{graylight}{#1}\par}
{\end{leftrule}}

\NewDocumentEnvironment{problem}{O{\textbf{Problema:}}} % Define example environment
{\begin{leftrule}\noindent\textcolor{graylight}{#1}\par}
{\end{leftrule}}

\begin{document}

%\tableofcontents

\newpage

\subfile{teorica/clase_01.tex}

\newpage

\subfile{teorica/clase_02.tex}

\newpage

\subfile{teorica/clase_03.tex}

\newpage

\subfile{teorica/clase_04.tex}

\newpage

\subfile{teorica/clase_05.tex}

\newpage

\subfile{teorica/clase_06.tex}

\newpage

\subfile{teorica/clase_07.tex}

\newpage

\subfile{teorica/clase_08.tex}

\newpage

\subfile{teorica/clase_09.tex}

\newpage

\subfile{teorica/clase_10.tex}

\newpage

\subfile{teorica/clase_11.tex}

\newpage

\subfile{teorica/clase_12.tex}

\newpage

\subfile{teorica/clase_13.tex}

\newpage

\subfile{teorica/clase_14.tex}

\newpage

\subfile{teorica/clase_15.tex}

\newpage

\subfile{teorica/clase_16.tex}

\newpage

\subfile{teorica/clase_17.tex}

\newpage

\subfile{teorica/clase_18.tex}

\newpage

\subfile{teorica/clase_19.tex}

\newpage

\subfile{teorica/clase_20.tex}

\newpage

\subfile{teorica/clase_21.tex}

\newpage

\subfile{teorica/clase_22.tex}



\end{document}