\documentclass{amsart}
\usepackage[paper=a4paper, left=1.5cm, right=1.5cm, bottom=1.5cm, top=1.5cm]{geometry}

\usepackage{amssymb}
\usepackage{dsfont}
\usepackage{amsmath}
\usepackage{amssymb,latexsym}
\usepackage{graphicx}
\usepackage{tabto}

\newtheorem*{teorema}{Theorem}
\newtheorem*{lema}{Lemma}
\newtheorem*{definicion}{Definicion}
\newtheorem*{observacion}{Observacion}

\renewcommand\qedsymbol{$\blacksquare$}

%clase de consulta la semana del parcial, se propuso un martes a las 17:30, entra la practica 1, 2, 3 y 4


%sudo apt-get install xzdec
%tlmgr init-usertree
%tlmgr install xzdec
%tlmgr install fontspec
%\usepackage{fontspec}
\usepackage{mdframed} % Add easy frames to paragraphs
\usepackage{lipsum} % For dummy text
\usepackage{xcolor}
\usepackage{xparse} % Add support for \NewDocumentEnvironment
\definecolor{graylight}{cmyk}{.30,0,0,.67} % define color using xcolor syntax

\newmdenv[ % Define mdframe settings and store as leftrule
linecolor=graylight,
topline=false,
bottomline=false,
rightline=false,
skipabove=\topsep,
skipbelow=\topsep
]{leftrule}

\NewDocumentEnvironment{ejercicio}{O{\textbf{Ejercicio:}}} % Define example environment
{\begin{leftrule}\noindent\textcolor{graylight}{#1}\par}
{\end{leftrule}}

\begin{document}

\section*{Clase 05}

\begin{ejercicio}
\begin{enumerate}
	\item[a)] $\Gamma \models \varphi \iff \Gamma \cup \{ \neg \varphi \}$ es insatisfacible
	\item[b)] $\theta$ es satisfacible $\iff$ $\theta \neg \models \varphi$
\end{enumerate}

\begin{proof}
Parte a)

$\Rightarrow$ Sea $\mathcal{I}$ una interpretacion, $\models_{I} \Gamma$, debe ser $\models_{I} \varphi$. Asi que $\nvDash_{I} \neg \varphi$. Por lo tanto $\underbrace{\nvDash_{I} \Gamma \cup \{  \neg \varphi\}}_{\textbf{insatisfacible}}$

$\Leftarrow$ Sea $\mathcal{I}$ una interpretacion, $\models_{I} \Gamma$. Como $\neg \varphi^{I}$ es falso, la negaci\'on es verdadera $\models_{I} \varphi^{I}$
\end{proof}

\begin{proof}
Parte b)

$\Delta$ es \textbf{satisfacible} $\iff \Delta \nvDash \varphi$, para todo $\psi$ es \textbf{insatisfacible}

$\Rightarrow$ Si $\mathcal{I}$ es una interpretaci\'on tal que $\models_{I} \Delta$ no puede ser $\models_{I} \psi$

$\Leftarrow$ Para probar esto, utilizo el contrarrec\'iproco. El \'unico ejemplo que dio fue el siguiente "Si llueve, entonces el patio est\'a mojado" su contra rec\'iproco es: "si el patio est\'a seco, entonces no llovi\'o", dicho esto, solamente hizo la suposici\'on \textbf{Supongo que $\Delta$ es insatisfacible.\textbf{•}}

\end{proof}

\end{ejercicio}

\begin{ejercicio}
Son equivalentes:

\begin{enumerate}
	\item[1)] $\Gamma \models \psi \rightarrow \Gamma \vdash \psi$
	\item[2)] Cualquier conjunto consistente de f\'ormulas es satisfacible
\end{enumerate}


\begin{proof}
1) $\Rightarrow$ 2).

Sea $\Delta$ \textbf{insatisfacible}. $\Delta = \Gamma \cup	\{ \Gamma\}$. Observaci\'on: $(\Gamma = \Delta / \{ \varphi\})$. Entonces $\Gamma \models \neg \varphi$.

Pero por 1), $\Gamma \vdash \neg \varphi$. Entonces $\Delta \vdash \neg \varphi$.

Por otro lado tenemos que:

\begin{align*}
\varphi &\in \Delta	\\
\varphi &\rightarrow \varphi \\
\Delta &\vdash \varphi
\end{align*}

Ahora nos quedo $\Delta \vdash \neg \varphi$ y $\Delta \vdash \varphi$. Ahora vemos que $\Delta$ es inconsistente.
\end{proof}

\begin{proof}
2) $\Rightarrow$ 1).

Sea $\Gamma$ y $\varphi$, tal que $\Gamma \models \varphi$. Por la parte a), $\Gamma \cup \{ \neg \varphi\}$ es \textbf{insatisfacible}. Por 2), tenemos que $\Gamma \cup \{ \neg \varphi \}$ es inconsistente. Por el teorema del absurdo, $\Gamma \vdash \varphi$
\end{proof}

\end{ejercicio}

\subsection*{Teorema de Compacidad}

\begin{enumerate}
	\item[a)] S\'i $\Gamma \models \varphi$, entonces $\exists \Gamma_0 \subseteq \Gamma$ finito tal que $\Gamma_0 \models \varphi$
	
	\item[b)] Si todo subconjunto finito de $\Gamma_0 \subseteq \Gamma$ es satisfacible en $T$. $\Gamma$ es satisfacible.
\end{enumerate}

Consecuencia: Si $\Gamma$ tiene modelos arbitrariamente grandes, entonces tiene modelos infinitos.

\begin{ejercicio}
Dado un conjunto, tiene como mucho dos elementos:

$\lambda = \forall x_1, \forall x_2, \forall x_3 \; (x_1 = x_2) \lor (x_1 = x_3) \lor (x_2 = x_3)$
\end{ejercicio}

\begin{ejercicio}
Dado un conjunto, tiene al menos dos elementos:

$\varphi = \exists x_1, \exists x_2 \; \neg (x_1 = x_2)$
\end{ejercicio}

\begin{ejercicio}
Dado un conjunto, tiene exactamente dos elementos:

$\alpha = \lambda \land \varphi$

Moraleja: Divide y vencer\'as. No doy todas las propiedades que debe cumplir el conjunto, sino que las defino por separado y luego las concateno con un $\land$.
\end{ejercicio}

\begin{observacion}
Sea $\varLambda_0 \subset \varLambda$, finito:

$\varLambda_0 = \{ \varphi_{n_1}, \varphi_{n_2}, \ldots, \varphi_{n_k} \} \cup \Gamma_0$ con $(n_1 < n_2 < \ldots < n_k)$

Si $\mathcal{I}$ es un modelo de $\Gamma$ con al menos $\Gamma_k$ elementos, entonces lo es $\Gamma_0$ y tambi\'en de $\varLambda_0$.

Por la parte b) del teorema de compacidad, $\varLambda$ es satisfacible. As\'i cualquier modelo de $\varLambda$ ser\'ia un modelo infinito de $\Gamma$
\end{observacion}

\begin{ejercicio}
Sea $\mathcal{G}$ un grafo y sea $k \in \mathbb{N}$. $\mathcal{G}$ es un k-coloreable si y solo si todo subgrafo finito de $\mathcal{G}$ es coloreable.

\begin{proof}
$\;$

$\Rightarrow$) es trivial

$\Leftarrow$)

Definimos un lenguaje $\mathcal{L} = <\emptyset, \emptyset, \mathcal{E}^2, = >$

$\varphi = \forall x \forall y \underbrace{(x \mathcal{E} y \rightarrow y \mathcal{E} x)}_{\text{no es bidireccional}} \land \underbrace{ \forall x \neg (x \mathcal{E} x) }_{\text{no es reflexivo}}$

Ahora volvemos a definir nuestro lenguaje, ya que los vertices tienen colores, entonces nuestro lenguaje queda como:

$\mathcal{L} = <\text{vertices(G)}, \emptyset, \mathcal{E}, \underbrace{R_1, \ldots, R_k}_{\text{colores}}, = >$

Ya teniendo definido el lenguaje, empecemos a definir las propiedades que debe cumplir el grafo:

\begin{enumerate}
	\item Dos estan relacionados pero no tienen el mismo color.
	
	$\psi_i = (\forall x \; \forall y \; R_i(x) = R_i(y) \leftarrow \neg (x \; \mathcal{E} \; y)$ 
	
	$\psi = \psi_1 \land \psi_2, \ldots, \psi_k$
	
	\item Todos estan coloreados.
	
	$\rho = \forall x \; (R_1(x) \lor R_2(x) \lor \ldots \lor R_k(x))$
	
	\item Enumero a todos los elementos.
	
	$\varphi_n = \{ \text{Hay al menos n} \}$
	
\end{enumerate}
Con todo esto, ya podemos definir la siguiente modelo

$\Gamma = \{ \rho, \psi, \varphi_n \; n \in \mathbb{N} \} \cup \{ c \; \mathcal{E} \; d: \text{c y d estan unidos en G} \}$

Sea $\Gamma_0 \subseteq \Gamma$ finito. Es cierto que $\Gamma_0$ es satisfacible?. Si, agarra un subgrafo de $G$ de $n_k$ variables.

$\Gamma_0 = \{ \rho, \varphi_{n_1}, \psi, \ldots, \rho, \varphi_{n_k}, \psi \}$ con $n_1 < \ldots < n_k$

Por el teorema de compacidad, $\Gamma$ tiene al menos un modelo y este contiene al grafo G.

\end{proof}

\end{ejercicio}

\begin{definicion}
Sea $\mathcal{L}$ un conjunto de formulas $\Sigma$ es \textbf{decidible} si para toda formula $\varphi$ hay un procedimiento efectivo en el que decide si $\varphi \in \Sigma$
\end{definicion}

\begin{ejercicio}
$\Sigma$ es finito $\Rightarrow$ es decidible.

Halting Problem $\Rightarrow$ es no decidible.
\end{ejercicio}

\begin{teorema}
Un conjunto de formulas cerradas implicaciones. Una teoria es \textbf{completa} si $\forall \theta \; \theta \in \Sigma$ \'o $\neg \theta \in \Sigma$. Una teoria es \textbf{axiomatizable} si existe un A conjunto de formula tal que sus consecuencias $A = \Sigma$.
\end{teorema}

\begin{teorema}
Toda teoria finitamente axiomatizable y completa es decidible.
\end{teorema}

\begin{ejercicio}
Teoria de numeros:

Sea el lenguaje $\mathcal{L} = <1, \mathcal{S}^1, = >$ y sea la interpretacion $\mathcal{I} = (\mathbb{N}, 1, \emptyset)$

th(N) = conjunto de formulas de $\mathcal{L}$ que son validas en la interpretacion de $\mathbb{N}$

Axiomas:

\begin{enumerate}
	\item[s1)] $\forall x \; s(x) \neq 1$
	\item[s2)] $\forall x \; \forall y \; (s(x) = s(y) \rightarrow x = y)$
	\item[s3)] $\forall y \; (y \neq 1 \rightarrow \exists s(x) = y)$
	\item[s4)] $\forall x \; s(x) \neq x$
	
	$\forall \underbrace{s \ldots s(x) \neq x}_{\text{n veces}}$
\end{enumerate}


Ahora llamo a $\mathcal{A}$ al conjunto de axiomas. $\mathcal{A} = \{s_1, \ldots, s_{k_i}, \ldots, s_{k_n}\}$


$conj(A) \subseteq th(N)$


\end{ejercicio}

Como son los modelos de Conj(A)?

\begin{enumerate}
	\item Necesita tener un primer elemento $d$ y necesitas $\underbrace{d \rightarrow S(d) \rightarrow S^2(d) \rightarrow S^3(d)}_{\textbf{tienes elementos distintos}}$
	\item Si tengo otro valor $\alpha$ y aplico la misma logica $\underbrace{\alpha \rightarrow S(\alpha) \rightarrow S^2(\alpha) \rightarrow S^3(\alpha)}_{\textbf{genero otra cadena, no es la misma}}$
\end{enumerate}

Modelo(Con(A)) = Es algo conocido. Entiendo el modelo.

Si uno conoce los modelos de una teoria, podemos conocer una teoria.

Teoria: $\underbrace{Con(A)}_{\textbf{consecuencias}}$ es completa

\begin{ejercicio}
$\underbrace{Con(A)}_{\textbf{completas}} = \underbrace{thN}_{\textbf{satisfacible}}$

$Con(A) \subseteq thN$

$\theta \in thN$ entonces $\theta \in Con(A)$ o $\theta \notin Con(A)$ (porque es completo, o esta o no esta)

Se que $\theta \in Con(A)$ o $\theta \notin Con(A)$. Si $\neg \theta \in Con(A), A \models \neg \theta$ entonces $\neg \theta \in Con(A) \subseteq thN$ y $thN$ es consistente, porque es satisfacible.

\end{ejercicio}

\end{document}



