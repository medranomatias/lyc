\section{Ejercicio:}

Sea \calpha una formula que no contiene el simbolo \cneg.
\begin{enumerate}
	\item Probar que \length(\calpha) es impar
	\item Probar que mas de un cuerto de los simbolos son variables.
\end{enumerate}

\subsection*{Ejercicio 1} 

Observaciones: Sea p\subi{n} = $\underbrace{p ||| \ldots}_{n}$. Se considera a la longitud como $\ell(p_n) = n + 1$, se define la \textbf{longitud corregida} como $\ell^*(p_n) = 1$.

Ejemplo: 

\calpha = ((p\subi{1} \cland p\subi{2}) \then p\subi{3})

\length(\calpha) = 1 + 1 + 1 + 1 + 2 + 1 + 3

\length\supi{*}(\calpha) = 1 + 1 + 1 + 1 + 1 + 1 + 1

\subsubsection*{Induccion sobre la complejidad}

P(n) = "Si \calpha es una palabra de complejidad n, entonces su largo es impar".

P(0) = Sea \calpha de complejidad 0 \then es una variable. Como \calpha resulta ser una variable, tiene largo 1 que es impar.

P(n) \then P(n + 1) 

Asumamos que vale P(n) y queremos probar P(n + 1). Sea \calpha de complejidad n + 1 (sin negaciones), con \calpha = (P * Q).

Asumamos que vale P(k), \cforall k $\leq$ n y sabemos que vale $P(n + 1)$ sin negaciones como $\alpha = (P * Q)$ y $C(P), C(Q) \leq n$ por hipotesis inductiva decimos que:

\begin{align} 
	\ell(P) &\equiv 1 \pmod 2\\
	\ell(Q) &\equiv 1 \pmod 2
\end{align}

Sigue de aqui que:

\begin{align} 
	\ell(P) &\equiv 1 + \ell(P) + \ell(Q) + 1 \pmod 2\\
	\ell(P) &\equiv 1 \pmod 2
\end{align}

El principio de induccion completa asegura que $P(n)$ es verdadera para todo n.

\subsection*{Ejercicio 2} 

P(n) = "Si $\alpha$ no tiene $\neg$ y tiene $\#var(\alpha) = n$, entonces $\ell(\alpha) = 4\times(n - 1) + 1$

P(0) es verdadero.

