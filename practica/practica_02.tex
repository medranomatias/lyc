\documentclass{amsart}
\usepackage[paper=a4paper, left=1.5cm, right=1.5cm, bottom=1.5cm, top=1.5cm]{geometry}


\usepackage{amssymb}
\usepackage{dsfont}
\usepackage{amsmath}
\usepackage{amssymb,latexsym}
\usepackage{graphicx}
\usepackage{tabto}

\newtheorem{theorem}{Theorem}
\newtheorem{lemma}{Lemma}
\newtheorem{definition}{Definicion}
\newtheorem{observation}{Observacion}

\renewcommand\qedsymbol{$\blacksquare$}

%sudo apt-get install xzdec
%tlmgr init-usertree
%tlmgr install xzdec
%tlmgr install fontspec
%\usepackage{fontspec}
\usepackage{mdframed} % Add easy frames to paragraphs
\usepackage{lipsum} % For dummy text
\usepackage{xcolor}
\usepackage{xparse} % Add support for \NewDocumentEnvironment
\definecolor{graylight}{cmyk}{.30,0,0,.67} % define color using xcolor syntax

\newmdenv[ % Define mdframe settings and store as leftrule
  linecolor=graylight,
  topline=false,
  bottomline=false,
  rightline=false,
  skipabove=\topsep,
  skipbelow=\topsep
]{leftrule}

\NewDocumentEnvironment{example}{O{\textbf{Ejemplo:}}} % Define example environment
{\begin{leftrule}\noindent\textcolor{graylight}{#1}\par}
{\end{leftrule}}


\begin{document}

\section{Clase 04}

\subsection{Consecuencia Semantica}

Diremos que $\Gamma$ (conjunto de formulas) implica semanticamente a $\varphi$ si en toda interpretacion $\mathcal{I}$ tal que $\models_{I} \Gamma$ ($\iff \models_{I} \gamma[s] \forall \gamma \in \Gamma$ y $S \colon Var \rightarrow \mathcal{U}_{I}$)

Entonces $\models_{I} \varphi$ y se nota $\Gamma \models \varphi$

\begin{example}
$\Gamma = \text{"axiomas de grupos"}$ 

$\varphi = (\forall x, \forall y ((x \times y = y) \land (y \times x = y)) \rightarrow x = 1)$

$\Gamma \models \varphi$

Si $\Gamma = \{ \textit{"Axioma de grafos sin loops y no dirigidos"} \}$

$\varphi = \forall x, \exists y \; E(x, y) = \textit{"Graficos sin vertices aislados"}$

$\Gamma \models \varphi$

Pues $\{ \bullet \} \in \Gamma$ pero $\{ \bullet \} \notin \varphi$

$\Gamma^{'} = \textit{"Grafos conexos"}$

$\Gamma^{'} = \varphi$

$\Gamma^{'} = \emptyset \models \varphi$

\end{example}

\begin{definition}
Si $\emptyset \models \varphi$ decimos que $\varphi$ es \textbf{universalmente valida} (vale en cualquier interpretacion de $\mathcal{I}$)
\end{definition}

\begin{example}

\begin{enumerate}
	\item[$\bullet$] $\forall x (P(x) \rightarrow P(x))$
	\item[$\bullet$] $\forall x, \exists y (Q(x, y) \lor (\neg Q(x, y)))$
	\item[$\bullet$] $(\forall x P(x, y)) \rightarrow P(y, y)$ $\varphi$
\end{enumerate}

Demostracion de que es universalmente valida: 

$\mathcal{L}_{T} = <\emptyset, \emptyset, \mathcal{P}^{2}>$

$\mathcal{I} = <\mathcal{U}_{I}, \mathcal{P} \subseteq \mathcal{U}_{I}^{2}>$

$\varphi_{I}[s] = \forall x \; \mathcal{P}(x, u) \rightarrow \mathcal{P}(u, u)$

En donde $s \colon Var \rightarrow \mathcal{U}_{I}$ y $y \rightarrow u$ (reemplazo las variables libres por u).

Ahora nos queda como:

$\varphi[s] = (\forall x \in \mathcal{U}_{I} \quad (u, x) \in \mathcal{P} \rightarrow (u, u) \in \mathcal{P})$

$v_{I}(\forall x \in \mathcal{U}_{I} (x, u) \in \mathcal{P} \underbrace{\rightarrow}_{p \rightarrow q = \neg p \lor q} (u, u) \in \mathcal{P}) = max\{ v_{I}((u, u) \in \mathcal{P}); 1 - v_{I}(\forall x \in \mathcal{U}_{I}(x, u) \in \mathcal{P}) \} = 1$

Si $v_{I}((u, u) \in \mathcal{P}) = 0$ quiere decir que $(u, u) \notin \mathcal{P}$ entonces es falso $\forall x \in \mathcal{U}_{I}$ $(x, u) \in \mathcal{P}$. $\inf_{x \in \mathcal{U}_{I}} v_{I}((x, u) \in \mathcal{P}) = 0 \rightarrow 1 - v_{I} = 1$


\end{example}

\begin{theorem}
$\models \varphi \rightarrow \psi \iff \underbrace{\{ \varphi \} \models \psi}_{\textit{asi es mas facil quiza demostrar}}$ 
\end{theorem}

\begin{definition}
Un conjunto $\Gamma$ es \textbf{satisfacible} si existe una interpretacion $\mathcal{I}$ junto con una valuacion si $Var \rightarrow \mathcal{U}_{I}$ tal que $\models_{I} \Gamma[s]$

Un  conjunto es \textbf{insatisfacible} si no es \textbf{satisfacible}
\end{definition}

\begin{observation}
Si $\varphi$ es universalmente valido $\rightarrow \underbrace{\neg \varphi}_{\textbf{valor de verdad} = 0}$ es insatisfacible

$\{ \varphi_1, \ldots, \varphi_n\} \models \psi \iff (\varphi_1 \rightarrow (\varphi_2 \rightarrow (\ldots (\varphi_n \rightarrow \psi))$
\end{observation}

\begin{definition}
Una deduccion de $\psi$ a partir de un conjunto $\Gamma$ es una cadena o una sucesion de formulas $\varphi_1, \ldots, \varphi_n$ con $\varphi_n = \psi$

\begin{enumerate}
	\item[$\bullet$] $\varphi_i$ es un elemento de $\Gamma$
	\item[$\bullet$] $\varphi_i$ es un axioma logico
	\item[$\bullet$] $\varphi_i$ se obtiene por M.P., existen $\varphi_j, \varphi_k$ con $j, k < i$ / $\varphi_j = \varphi_k \rightarrow \varphi_i$	
	\item[$\bullet$] $\varphi_i$ es una generalizacion; existen $\varphi_j, j < i$ y una variable x / $\varphi_i \; \forall x \; \varphi_j$
\end{enumerate}

$\Gamma \vdash \psi$

\begin{enumerate}
	\item[A1] $\varphi \rightarrow (\psi \rightarrow \varphi)$
	\item[A2] $(\varphi \rightarrow (\psi \rightarrow \rho)) \rightarrow ((\varphi \rightarrow \psi) \rightarrow (\varphi \rightarrow \rho))$
	\item[A3] $(\neg \psi \rightarrow \neg \varphi) \rightarrow [(\neg \psi \rightarrow \varphi) \rightarrow \psi]$
	\item[A4] $\forall x \; (\varphi \rightarrow \psi) \rightarrow (\varphi \rightarrow \forall x \; \psi)$ si x no es libre en $\varphi$
	\item[A5] $\forall x \; \varphi \rightarrow \varphi(x / t)$ con \textbf{t} libre para \textbf{x}. Cuando reemplazo \textbf{x} por \textbf{t} las variables \textbf{t} no se cuantifican.
\end{enumerate}

\begin{align*}
\neg \neg p &\iff p \\
p \lor q &\iff (\neg p \rightarrow q) \\
p \land q &\iff \neg (p \rightarrow \neg q) \\
\exists x \; \varphi &\iff \neg \forall x \; \neg \varphi
\end{align*}

\begin{example}

\end{example}

\end{definition}

\end{document}


